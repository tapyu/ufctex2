\chapter{Introdução}
\label{cap:introducao}

\begin{table}[h!]	
	\centering
	\Caption{\label{tab:label_da_tabela} Legenda da Tabela}
	\UECEtab{}{
		\begin{tabular}{ccll}
			\toprule
			Quisque & pharetra & tempus & vulputate \\
			\midrule \midrule
			E1 & Complete coverage & Both splice sites \\
			E2 & Complete coverage & Both splice sites \\
			E3 & Partial coverage & Both splice sites & Both \\
			E4 & Partial coverage & One splice site & Both \\
			E5 & Complete or coverage & No splice & Both \\
			E6 & No coverage & No splice sites\\
			\bottomrule
		\end{tabular}
	}{
		\Fonte{Elaborado pelo autor}
	}
\end{table}

\begin{quadro}[h!]	
	\centering
	\Caption{\label{qua:label_do_quadro} Legenda do Quadro}
	\UECEqua{}{
		\begin{tabular}{|c|c|}
			\hline
			Quisque & pharetra \\
			\hline
			E1 & Complete coverage  \\
			\hline
			E2 & Complete coverage \\
			\hline
		\end{tabular}
	}{
		\Fonte{Elaborado pelo autor}
	}
\end{quadro}

\begin{figure}[h!]
	\centering
	\Caption{\label{fig:label_da_figura} Legenda da Figura}	
	\UECEfig{}{
		\includegraphics[width=8cm]{figuras/figura-1.jpg}
	}{
		\Fonte{Elaborado pelo autor}
	}	
\end{figure}	